\documentclass[preprint,russian,a5paper,10pt,twoside]{ncc}
\usepackage[utf8]{inputenc}
\usepackage{misccorr}			% различные особенности оформления документов, принятые в России
\usepackage{indentfirst}		% начинать абзац с красной строки
\usepackage{tikz}				% рисование чертежей
\usetikzlibrary{positioning}	% автоматическое размещение элементов на чертежах относительно друг друга
\usetikzlibrary{shapes.misc}
\begin{document}

\begin{titlepage}
	\titlehead{\footnotesize САНКТ-ПЕТЕРБУРГСКИЙ\\[-.2ex] ГОСУДАРСТВЕННЫЙ ПОЛИТЕХНИЧЕСКИЙ УНИВЕРСИТЕТ\par\vspace{1mm} ИНСТИТУТ ИНФОРМАЦИОННЫХ ТЕХНОЛОГИЙ И УПРАВЛЕНИЯ}
	\titlefoot{Санкт-Петербург\\ \theyear}
	\author{С.\,А.~Нестеров}
	\title{ОПТИМАЛЬНЫЕ СИСТЕМЫ УПРАВЛЕНИЯ}
	\titlecomment{Конспект лекций}
	\maketitle
\end{titlepage}

\setcounter{page}{2}
\thispagestyle{empty}
\mbox{}\newpage
\tableofcontents
\newpage

\section{Введение\label{intro}}

Существует три фундаментальных принципа управления:
\begin{itemize}
\item Управление по заданию --- передаточная функция регулятора обратна передаточной функции объекта управления (ОУ).
\begin{equation}\label{equ:intro:control_by_target}
W_{\text{УУ}}=\frac{1}{W_{\text{ОУ}}}
\end{equation}
%\begin{figure}[ht] \centering
%\begin{picture}(60,10)							% по умолчанию размеры в мм
%	\put(0,5){\vector(1,0){10}}
%	\put(0,5){\makebox(10,6){$v$}}
%	\put(10,0){\framebox(15,10){$ W_{\text{УУ}} $}}
%	\put(25,5){\vector(1,0){10}}
%	\put(25,5){\makebox(10,6){$u$}}
%	\put(35,0){\framebox(15,10){$ W_{\text{ОУ}} $}}
%	\put(50,5){\vector(1,0){10}}
%	\put(50,5){\makebox(10,6){$y$}}
%\end{picture}
%\footnotesize \caption{Управление по заданию\label{fig:intro:control_by_target}}
%\end{figure}

\begin{figure}[ht] \centering
\begin{tikzpicture}											% здесь размеры по умолчанию в см
	[	auto,												% чтобы элементы чертежа не пересекались
		inner sep = 2ex,									% пространство, оставляемое вокруг текста
		block/.style={rectangle,draw=black,thick},		% стиль отрисовки блока
		vec/.style={->,>=stealth,semithick},				% стиль отрисовки вектора
		point/.style={inner sep = 0pt}	]
	\node[block] (object) {$ W_{\text{ОУ}} $};			% Ставить ; обязательно! ИНАЧЕ ВСЁ ВИСНЕТ!!!!1
	\node[block] (regulator) [left = of object] {$ W_{\text{УУ}} $};
%	\draw[vec] (regulator) to (object);					% вектор без подписи
	\draw[vec] (regulator) to node{$u$} (object);			% вектор с подписью
	\node[point] (input) [left = of regulator] {};
	\draw[vec] (input) to node{$v$} (regulator);
	\node[point] (output) [right = of object] {};
	\draw[vec] (object) to node{$y$} (output);
\end{tikzpicture}
\footnotesize \caption{Управление по заданию\label{fig:intro:control_by_target}}
\end{figure}

\item Управление по возмущению --- возмущения, действующие на объект, учитываются при формировании управляющего воздействия.
\begin{equation}\label{equ:intro:control_by_noise}
W_{\text{УУ}}=?
\end{equation}
\begin{figure}[ht] \centering
\begin{tikzpicture}	
	[	auto, inner sep = 2ex,
		block/.style={rectangle,draw=black,thick},
		vec/.style={->,>=stealth,semithick},
		point/.style={inner sep = 0pt}	]
	\node[block] (object) {$ W_\text{ОУ} $};
	\node[block] (regulator) [left = of object] {$ W_\text{УУ} $};
	\draw[vec] (regulator) to node{$u$} (object);
	\node[point] (input) [left = of regulator] {};
	\draw[vec] (input) to node{$v$} (regulator);
	\node[point] (output) [right = of object] {};
	\draw[vec] (object) to node{$y$} (output);
	\node[point] (noise) [above = of object] {};
	\draw[vec] (noise) to node{$f$} (object);
\end{tikzpicture}
\footnotesize \caption{Управление по возмущению\label{fig:intro:control_by_noise}}
\end{figure}

\item Управление по ошибке --- на вход регулятора подаётся отклонение выходного сигнала объекта от сигнала задания, и формируется управляющий сигнал, направленный на минимизацию этого отклонения.
\begin{equation}\label{equ:intro:control_by_eps}
W_{\text{УУ}}=?
\end{equation}
\begin{figure}[ht] \centering
\begin{tikzpicture}	
	[	auto, inner sep = 2ex,
		block/.style={rectangle,draw,thick},
		vec/.style={->,>=stealth,semithick},
		point/.style={inner sep = 0pt}	]
	\node[block] (object) {$ W_\text{ОУ} $};
	\node[block] (regulator) [left = of object] {$ W_\text{УУ} $};
	\draw[vec] (regulator) to node{$u$} (object);
	\node[circle,draw,thick,inner sep=0pt,minimum size=2ex] (sum) [left = of regulator] {};
	\draw[semithick] (sum.north west) -- (sum.south east);
	\draw[semithick] (sum.south west) -- (sum.north east);
	\fill (sum.center) -- (sum.south west) arc (-135:-45:1ex) -- cycle;
	\draw[vec] (sum) to node{$e$} (regulator);
	\node[point] (input) [left = of sum] {};
	\draw[vec] (input) to node{$v$} (sum);
	\node[point] (output) [right = of object] {};
	\draw[vec] (object) to node(y){$y$} (output);
	\node[point] (p1) [below=of sum] {};
	\draw (y.south) |- (p1.center) [vec] to (sum);
	\fill (y.south) circle (1pt);
	\node[point] (noise) [above = of object] {};
	\draw[vec] (noise) to node{$f$} (object);
\end{tikzpicture}
\footnotesize \caption{Управление по ошибке\label{fig:intro:control_by_eps}}
\end{figure}
\end{itemize}
% Смотри формулу~\eqref{equ:intro:control_by_target}, с.~\pageref{equ:intro:control_by_target}.
% А теперь смотри рисунок~\ref{fig:intro:control_by_target}, с.~\pageref{fig:intro:control_by_target}.

\newpage\mbox{}\newpage

\section{Постановка задачи оптимального управления\label{task}}
В общем случае такой постановки управления характеризуются значительным разнообразием, что связано с 3 основными причинами:

\begin{itemize}
\item Конкретным условиям функционирования, к которым следует отнести - начальное(исходное) и конечное (требуемое) состояние ОУ с временной привязкой при \[{{X}_{0}}\left( {{t}_{0}} \right)\], \[{{X}_{0}}\left( {{t}_{}} \right)\] - взаимосвязи присущие реальным процессам в форме алгебраических (голономных) уравнений $G\left( x,u \right)=0$ дифференциальных (неголономных) $\varphi \left( x,u,t \right)=\dot{x}-Ax+BV=0$ и интегральных (изопериметрических) $I\left( x,u,t \right)=\int\limits_{{{t}_{0}}}^{{{t}_{}}}{\varphi \left( x,u,t \right)dt=C}$;
\item Ограничения на управляющие сигналы и переменные состояния .
\end{itemize}
\par

\end{document}




