\documentclass[preprint,russian,a5paper,10pt,twoside]{ncc}
\usepackage[utf8]{inputenc}
\usepackage{misccorr}			% различные особенности оформления документов, принятые в России
\usepackage{indentfirst}		% начинать абзац с красной строки
\usepackage{tikz}				% рисование чертежей
\usetikzlibrary{positioning}	% автоматическое размещение элементов на чертежах относительно друг друга
\usetikzlibrary{shapes.misc}
\usepackage{enumitem} %Для разрывов в списках

\begin{document}

\begin{titlepage}
	\titlehead{\footnotesize САНКТ-ПЕТЕРБУРГСКИЙ\\[-.2ex] ГОСУДАРСТВЕННЫЙ ПОЛИТЕХНИЧЕСКИЙ УНИВЕРСИТЕТ\par\vspace{1mm} ИНСТИТУТ ИНФОРМАЦИОННЫХ ТЕХНОЛОГИЙ И УПРАВЛЕНИЯ}
	\titlefoot{Санкт-Петербург\\ \theyear}
	\author{С.\,А.~Нестеров}
	\title{ОПТИМАЛЬНЫЕ СИСТЕМЫ УПРАВЛЕНИЯ}
	\titlecomment{Конспект лекций}
	\maketitle
\end{titlepage}

\setcounter{page}{2}
\thispagestyle{empty}
\mbox{}\newpage
\tableofcontents
\newpage

\section{Введение\label{intro}}

Существует три фундаментальных принципа управления:
\begin{itemize}
\item Управление по заданию --- необходимо получить заданный сигнал $v$ на выходе объекта управления (ОУ). Для этого передаточная функция системы управления в целом (см. рис.~\ref{fig:intro:control_by_target}) должна быть равна единице. Тогда передаточная функция устройства управления (УУ) равна:
\begin{equation}\label{equ:intro:control_by_target}
W_{\text{УУ}}=\frac{1}{W_{\text{ОУ}}}
\end{equation}

\begin{figure}[ht] \centering		% [ht] - размещение в указанном месте, а если невозможно, то вверху следующей страницы
\begin{tikzpicture}											% здесь размеры по умолчанию в см
	[	auto,												% чтобы элементы чертежа не пересекались
		inner sep = 2ex,									% пространство, оставляемое вокруг текста
		block/.style={rectangle,draw=black,thick},		% стиль отрисовки блока
		vec/.style={->,>=stealth,semithick},				% стиль отрисовки вектора
		point/.style={inner sep = 0pt}	]
	\node[block] (object) {$ W_{\text{ОУ}} $};			% Ставить ; обязательно! ИНАЧЕ ВСЁ ВИСНЕТ!!!!1
	\node[block] (regulator) [left = of object] {$ W_{\text{УУ}} $};
%	\draw[vec] (regulator) to (object);					% вектор без подписи
	\draw[vec] (regulator) to node{$u$} (object);			% вектор с подписью
	\node[point] (input) [left = of regulator] {};
	\draw[vec] (input) to node{$v$} (regulator);
	\node[point] (output) [right = of object] {};
	\draw[vec] (object) to node{$x$} (output);
\end{tikzpicture}
\footnotesize \caption{Управление по заданию\label{fig:intro:control_by_target}}
\end{figure}

\item Управление по возмущению --- возмущения $f$, действующие на объект, учитываются при формировании управляющего воздействия $u$ (см. рис.~\ref{fig:intro:control_by_noise}). Передаточная функция от возмущений к выходу может быть выражена как $ W_{fu}W_\text{ОУ}+W_{fx} $ и должна быть равна нулю. Тогда передаточная функция от возмущений к управлению равна:
\begin{equation}\label{equ:intro:control_by_noise}
W_{fu}=-\frac{W_{fx}}{W_\text{ОУ}}
\end{equation}

\begin{figure}[ht] \centering
\begin{tikzpicture}	
	[	auto, inner sep = 2ex,
		block/.style={rectangle,draw=black,thick},
		vec/.style={->,>=stealth,semithick},
		point/.style={inner sep = 0pt}	]
	\node[block] (object) {$ W_\text{ОУ} $};
	\node[block] (regulator) [left = of object] {$ W_\text{УУ} $};
	\draw[vec] (regulator) to node{$u$} (object);
	\node[point] (input) [left = of regulator] {};
	\draw[vec] (input) to node{$v$} (regulator);
	\node[point] (output) [right = of object] {};
	\draw[vec] (object) to node{$x$} (output);
	\node[point] (noise) [above = of object] {};
	\draw[vec] (noise) to node(f){$f$} (object);
	\draw[vec] (f.west) -| (regulator.north);
	\fill (f.west) circle (1pt);
\end{tikzpicture}
\footnotesize \caption{Управление по возмущению\label{fig:intro:control_by_noise}}
\end{figure}

\item Управление по ошибке --- на вход регулятора подаётся отклонение $e$ выходного сигнала объекта $x$ от сигнала задания $v$ (см. рис.~\ref{fig:intro:control_by_eps}). Если $e$ устремить к нулю, то передаточная функция от $e$ к $x$ должна устремиться к бесконечности, чтобы получить на выходе заданную величину:
\begin{equation}\label{equ:intro:control_by_eps}
W_\text{УУ}W_\text{ОУ}\longrightarrow\frac{x_\text{уст}}{0}=\infty
\end{equation}
Этого можно добиться, например, введением в регулятор очень больших коэффициентов усиления или интегрирующих звеньев.

\begin{figure}[ht] \centering
\begin{tikzpicture}	
	[	auto, inner sep = 2ex,
		block/.style={rectangle,draw,thick},
		vec/.style={->,>=stealth,semithick},
		point/.style={inner sep = 0pt}	]
	\node[block] (object) {$ W_\text{ОУ} $};
	\node[block] (regulator) [left = of object] {$ W_\text{УУ} $};
	\draw[vec] (regulator) to node{$u$} (object);
	\node[circle,draw,thick,inner sep=0pt,minimum size=2ex] (sum) [left = of regulator] {};
	\draw[semithick] (sum.north west) -- (sum.south east);
	\draw[semithick] (sum.south west) -- (sum.north east);
	\fill (sum.center) -- (sum.south west) arc (-135:-45:1ex) -- cycle;
	\draw[vec] (sum) to node{$e$} (regulator);
	\node[point] (input) [left = of sum] {};
	\draw[vec] (input) to node{$v$} (sum);
	\node[point] (output) [right = of object] {};
	\draw[vec] (object) to node(x){$x$} (output);
	\node[point] (p1) [below=of sum] {};
	\draw (x.south) |- (p1.center) [vec] to (sum);
	\fill (x.south) circle (1pt);
\end{tikzpicture}
\footnotesize \caption{Управление по ошибке\label{fig:intro:control_by_eps}}
\end{figure}
\end{itemize}
% Смотри формулу~\eqref{equ:intro:control_by_target}, с.~\pageref{equ:intro:control_by_target}.
% А теперь смотри рисунок~\ref{fig:intro:control_by_target}, с.~\pageref{fig:intro:control_by_target}.

По решаемой задаче системы автоматического управления (САУ) можно разделить на системы:
\begin{itemize}
\item стабилизации
\item программного управления
\item слежения
\item оптимального управления
\item адаптивного управления (поисковые и самонастраивающиеся)
\end{itemize}

Анализ и синтез САУ предполагает, что кроме цели рассматривается путь решения задачи. Для этого используются различные показатели качества:
\begin{itemize}
\item Прямые:
	\begin{itemize}
	\item Статические: статизм, добротность;
	\item Динамические: время переходного процесса, перерегулирование, колебательность;
	\end{itemize}
\item Косвенные:
	\begin{itemize}
	\item Корневые;
	\item Частотные;
	\item Интегральные.
	\end{itemize}
\end{itemize}

Существует два основных подхода к синтезу САУ:
\begin{enumerate}
\item Обеспечить заданные показатели качества;
\item Обеспечить наилучшее (оптимальное) в каком-либо определённом смысле качество управления для заданного объекта при конкретных условиях работы и ограничениях.
\end{enumerate}

При решении такой задачи естественно требовать от управляемого процесса (ОУ) выполнения условий полной управляемости и наблюдаемости.

\newpage % \mbox{}\newpage

\section{Постановка задачи оптимального управления\label{task}}

%%Текст из пдф
%Сначала напоминания о фундаментальных принципах управления, вытекающих из них способах формирования управления и %структур систем:
%Задачи, решаемые системой:
%Цель: 
%-стабилизации
%-програмного движения
%-слежения. 
%Анализ и синтез СУ предполагает, что кроме цели (что?) рассматриваеться путь (как?) решения задачи.
%Существует 2 основных подхода:
%1). Обеспечить выполнение прямых (или косвенных) показателей качества управляемых процессов (точности- определяемой %ошибками; быстродействием-tпп., tдостижения - гладкости - пререгулированием, колебательностью; работопригодности - %запасы устнойчивости, малочуствительность, грубость)
%2). Управление формируеться так, чтобы достигать наилучшего (оптимального) в каком-либо определенном смыс

В общем случае такой постановки управления характеризуются значительным разнообразием, что связано с 3 основными причинами:



\begin{itemize}
\item Конкретным условиям функционирования, к которым следует отнести - начальное(исходное) и конечное (требуемое) состояние ОУ с временной привязкой при $X_{0}\left ( t_{0} \right )$ , $X_{k}\left ( t_{k} \right )$ 
\item Взаимосвязями присущими реальным процессам в форме алгебраических (голономных) уравнений $G\left( x,u \right)=0$, дифференциальных (неголономных) $\varphi \left( x,u,t \right)=\dot{x}-Ax+BV=0$ и интегральных (изопериметрических) $I\left( x,u,t \right)=\int\limits_{{{t}_{0}}}^{{{t}_{}}}{\varphi \left( x,u,t \right)dt=C}$;
\item Ограничения на управляющие сигналы и переменные состояния $\underline{U}\le U\le \overline{U}$; $\underline{X}\le X\le \overline{X}$; $\sum\limits_{i=1}^{n}{u_{i}^{2}}\le A$
\item Математическим видом критерия оптимальности. Выбор критерия не является формальным актом, он не предписываться какой-либо теоремой, а полностью определяется содержанием задачи управления. В общем виде критерий записывается функционалом: 
\[J\left( v,x,\underset{{{t}_{0}}-{{t}_{k}}}{\mathop{T}}\, \right)=\Phi \left( \underset{{{x}_{0}},{{x}_{k}}}{\mathop{x\left( T \right)}}\,,T \right)+\int\limits_{0\left( {{t}_{0}} \right)}^{T\left( {{t}_{k}} \right)}{F\left( x\left( t \right),v\left( t \right),t \right)}\]где $\Phi \left( \underset{{{x}_{0}},{{x}_{k}}}{\mathop{x\left( T \right)}}\,,T \right)$ - терминальная составляющая, характеризующая качество управления только по начальному и конечному состоянию ОУ (поэтому отсутствует $U$); $F\left( x\left( t \right),v\left( t \right),t \right)$ - подынтегральная целевая функция,характеризующая качество внутри интервала управления. $U$ управления должно доставлять $U=\underset{U}{\mathop{ext}}\,\left( J\left( x,v,T \right) \right)$. Вид критерия задает класс задач оптимальности: 
\begin{enumerate}
\item Критерий максимального быстродействия (минимального времени реакции). Варианты записи:
\begin{itemize}
\item $\Phi \left( \underset{{{x}_{0}},{{x}_{k}}}{\mathop{x\left( T \right)}}\,,T \right)=T={{t}_{k}}-{{t}_{0}}$,$F\left( x\left( t \right),v\left( t \right),t \right)\equiv 0$  
\item $\Phi \left( \underset{{{x}_{0}},{{x}_{k}}}{\mathop{x\left( T \right)}}\,,T \right)\equiv 0$,$F\left( x\left( t \right),v\left( t \right),t \right)=1\to \int\limits_{0}^{T}{1dx}=T={{t}_{k}}-{{t}_{0}}$
\end{itemize}
\item Критерий минимального расхода управления (топлива): 
\\\begin{math}\min J\left( {{t}_{k}}..{{t}_{0}} \right)\end{math}; $\Phi \left( \underset{{{x}_{0}},{{x}_{k}}}{\mathop{x\left( T \right)}}\,,T \right)\equiv 0$,$F\left( x\left( t \right),v\left( t \right),t \right)=R\left| v \right|=\sum\limits_{i=1}^{n}{{{r}_{i}}}\left| {{u}_{i}} \right|$
\item Критерий идеальной точности (минимальной ошибки):\\\begin{math}\Phi \left( \underset{{{x}_{0}},{{x}_{k}}}{\mathop{x\left( T \right)}}\,,T \right)\equiv 0\end{math}, \begin{math}F\left( x\left( t \right),v\left( t \right),t \right)={{x}^{T}}Qx\end{math}, где $x$ - как отклонение (ошибка) от заданного значения, $Q$-матрица весовых коэффициентов (важность каждой составляющей в общей ошибке).
\item Критерий минимума энергетических затрат:\\\begin{math}\Phi \left( \underset{{{x}_{0}},{{x}_{k}}}{\mathop{x\left( T \right)}}\,,T \right)\equiv 0\end{math}, \begin{math}F\left( x\left( t \right),v\left( t \right),t \right)={{U}^{T}}RU=\sum\limits_{i=1}^{n}{{{r}_{i}}}u_{i}^{2}\end{math} на интервале ${{t}_{k}}-{{t}_{0}}=T$
\item Критерий терминального управления (например задача мягкой посадки - система управления конечным положением):\\
$\Phi \left( \underset{{{x}_{0}},{{x}_{k}}}{\mathop{x\left( T \right)}}\,,T \right)\ne 0$, $F\left( x\left( t \right),v\left( t \right),t \right)=0$
\end{enumerate}
Some text
\begin{enumerate}[resume]
  \item Three
\end{enumerate}


\end{itemize}








\par

\end{document}




