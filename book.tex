\documentclass[preprint,russian,a5paper,10pt,twoside]{ncc}
\usepackage[utf8]{inputenc}
\usepackage{misccorr}		% различные особенности оформления документов, принятые в России
\usepackage{indentfirst}	% начинать абзац с красной строки
\begin{document}

\begin{titlepage}
	\titlehead{\footnotesize САНКТ-ПЕТЕРБУРГСКИЙ\\[-.2ex] ГОСУДАРСТВЕННЫЙ ПОЛИТЕХНИЧЕСКИЙ УНИВЕРСИТЕТ\par\vspace{3pt} ИНСТИТУТ ИНФОРМАЦИОННЫХ ТЕХНОЛОГИЙ И УПРАВЛЕНИЯ}
	\titlefoot{Санкт-Петербург\\ \theyear}
	\author{С.\,А.~Нестеров}
	\title{ОПТИМАЛЬНЫЕ СИСТЕМЫ УПРАВЛЕНИЯ}
	\titlecomment{Конспект лекций}
	\maketitle
\end{titlepage}

\setcounter{page}{2}
% \setcounter{tocdepth}{2}
\thispagestyle{empty}
\mbox{}
\newpage
\tableofcontents
\newpage

\section{Введение\label{intro}}

Существует три фундаментальных принципа управления:
\begin{itemize}
\item Управление по заданию --- передаточная функция регулятора обратна передаточной функции объекта управления.
\begin{equation}
W_{\text{УУ}}=\frac{1}{W_{\text{ОУ}}}
\end{equation}
\item Управление по возмущению --- возмущения, действующие на объект, учитываются при формировании управляющего воздействия.
\item Управление по ошибке --- на вход регулятора подаётся отклонение выходного сигнала объекта от сигнала задания, и формируется управляющий сигнал, направленный на минимизацию этого отклонения.
\end{itemize}

\newpage

\section{Постановка задачи оптимального управления\label{task}}
В общем случае такой постановки управления характеризуются значительным разнообразием, что связано с 3 основными причинами:

\begin{itemize}
\item Конкретным условиям функционирования, к которым следует отнести - начальное(исходное) и конечное (требуемое) состояние ОУ с временной привязкой при \[{{X}_{0}}\left( {{t}_{0}} \right)\], \[{{X}_{0}}\left( {{t}_{}} \right)\] - взаимосвязи присущие реальным процессам в форме алгебраических (голономных) уравнений $G\left( x,u \right)=0$ дифференциальных (неголономных) $\varphi \left( x,u,t \right)=\dot{x}-Ax+BV=0$ и интегральных (изопереметрических) $I\left( x,u,t \right)=\int\limits_{{{t}_{0}}}^{{{t}_{}}}{\varphi \left( x,u,t \right)dt=C}$;
\item Ограничения на управляющие сигналы и переменные состояния .
\end{itemize}
\par

\end{document}




