\documentclass[preprint,russian,a5paper,10pt,twoside]{ncc}
\usepackage[utf8]{inputenc}
\usepackage{misccorr}		% различные особенности оформления документов, принятые в России
\usepackage{indentfirst}	% начинать абзац с красной строки
\begin{document}
\begin{titlepage}
	\titlehead{\footnotesize САНКТ-ПЕТЕРБУРГСКИЙ\\[-.2ex] ГОСУДАРСТВЕННЫЙ ПОЛИТЕХНИЧЕСКИЙ УНИВЕРСИТЕТ\par\vspace{3pt} ИНСТИТУТ ИНФОРМАЦИОННЫХ ТЕХНОЛОГИЙ И УПРАВЛЕНИЯ}
	\titlefoot{Санкт-Петербург\\ \theyear}
	\author{С.\,А.~Нестеров}
	\title{ОПТИМАЛЬНЫЕ СИСТЕМЫ УПРАВЛЕНИЯ}
	\titlecomment{Конспект лекций}
	\maketitle
\end{titlepage}

\setcounter{page}{2}
% \setcounter{tocdepth}{2}
\thispagestyle{empty}
\mbox{}
\newpage
\tableofcontents
\newpage

\section{Введение\label{intro}}

О чём писать во введении? Или постановка задачи это и есть введение?

\newpage

\section{Постановка задачи оптимального управления\label{task}}

Фундаментальные принципы управления


\end{document}




